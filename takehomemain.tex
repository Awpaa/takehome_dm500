\documentclass{article}
\usepackage{graphicx} % Required for inserting images
\usepackage{amsfonts}

\title{Takehome}
\author{Alexander Almar, Andreas Jeppesen }
\date{November 2023}

\begin{document}

\maketitle

\section{Reeksamen februar 2015 opgave 1}
I det følgende lader vi \(U = \{1, 2, 3, . . . , 15\}\) være universet (universal set). Betragt de to mængder 
\(A = \{2n | n \in S\}\ og\ B = \{3n + 2 | n \in S\}\) hvor \(S = \{1, 2, 3, 4\}\). Angiv samtlige elementer i hver af følgende mængder

a) A

Ved at betragte mængden A kan man se at der multipliceres med 2 på hvert element i sættet U. Mængden A er derfor
\(\{2,4,6,8\}\)

b) B

Ved at betragte mængden B kan man se at der for hvert element i U multiplicieres med 3 og adderes med 2. Derfor vil mængden B være 
\(\{5,8,11,14\}\)

c) \(A \cap B\)

Ved at se om A og B har nogle elementer tilfælles kan man se at det har de ikke 
\(\{\}\)

d) \(A \cup B\)

Den samlede mængde for A og B er alle elemeter for A og for B lagt i 1 sæt. Svaret er 
\(\{2,4,5,6,8,10,12,14\}\)

e) \(A-B\)

De elementer som kun er i A og som ikke er i B er
\(\{2,4,6,8,10,12\}\)

f) \(\bar A \)

Alle andre elementer i sættet U end dem som er angivet i mænden A er 
\(\{1,3,4,7,9,11,13,15\}\)

\section{Reeksamen februar 2015 opgave 2}
a) Hvilke af følgende udsagn er sande?

\begin{enumerate}
    \item $\forall x \in\mathbb{N}: \exists y \in\mathbb{N}: x < y$
    \item $\forall x \in\mathbb{N}: \exists!y \in\mathbb{N}: x < y$
    \item $\exists y \in\mathbb{N}: \forall x \in\mathbb{N}: x < y$
\end{enumerate}

\begin{enumerate}
    \item Sandt, fordi for alle naturlige tal kan der findes et større
    \item Falsk, fordi der for alle naturlige tal findes en tællelig uendelig mængde større naturlige tal
    \item Falsk, da der ikke er et største naturligt tal
\end{enumerate}

\noindent b) Angiv negeringen af udsagn 1. fra spørgsmål a). Negerings-operatoren ($\neg$) må ikke indgå i dit udsagn.

$\exists x \in\mathbb{N}: \forall y \in\mathbb{N}: x > y$

\noindent Der negeres igennem hele udsagnet, så kvantorerne skiftes og det indre udsagn negeres.

\end{document}
